Since this is an ongoing research project, there is a long list of problems to address which were not investigated or reported in this thesis. The list below presents the most important of those.

\begin{itemize}
	\item Conduct a further investigation into the nature of the two strategies \achiever and \host, since the existing literature could not provide them with convincing strategies known to evolutionary psychology.
	\item Quantify how extreme individuals in each dataset are. What is the distribution of how many archetypes each individual is \textit{close} to, where \textit{close} is defined by some distance, such as a quarter of the average distance between archetypes. The analysis could be expanded to quantify the percentage of individuals with e.g. a certain job-type or education background are close to a particular archetype.
	\item In Section \ref{subsec:classification} it was shown there are six significant principal components in questionnaire inventory space and that these are better prediction targets than Big Five traits, in terms of increasing prediction accuracy. Personality archetypes are points in BF-space, but in light of this result could it be that they might rather belong in PC-space? This could be analyzed by collapsing multiple datasets and using PCA to find the common vectors of most explained variance, and then computing consensus archetypes similar to how it is explained in Section \ref{subsec:computingConsensusArchetypes} but in 5D PC-space (not 6 - recall Figure \ref{fig:convexHullCurseOfDimensionality}).
	\item Investigate whether the discovered archetypes are predicted with higher accuracy than equally extreme well-seperated random points on the convex hull.
	\item Employ a supervised basis-rotation optimization scheme that finds the projection of Big Five inventory items which provide the highest accuracy. Using this is an upper limit for how well the behavioral data can predict personality is found, and can be compared to the currently obtained accuracy for the archetypes.
	\item While the shape of performance functions of objectives has not been discussed in this thesis it has been of sporadic interest to this project to treat them as such. Sheftel explores this idea\mcite{sheftel2013geometry}, but considering Figure \ref{fig:objectivePerformance} one can imagine that the performance could have elliptic contours rather than circular. Using e.g. the standard deviations for each archetype trait as axes for the contours of the performance functions, shaped performance functions for each objective could be estimated and a more complex geometry of the emerging Pareto front could be computed.
	\item Based in the SensibleDTU dataset, an investigation into how archetype resemblance influences personal relationships is due. A hypothesis could be e.g. that the \hippie archetype should be observed in many work relationships with the \wildcard.
\end{itemize}