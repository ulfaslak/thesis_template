Evolution\ae r psykologi foresl\aa r at forskellige strategier for at udf\o re evolution\ae rt vigtige opgaver i livet har formet vores individuelle forskelle som mennesker. Men hvad disse strategier er har indtil videre v\ae ret g\ae tv\ae rk. Her anvender vi Pareto Task Inference til at udlede strategierne og deres adf\ae rdsm\ae ssige korrelater. Vi finder, at de samme seks personlighedsarketyper opst\aa r, p\aa  konsistent vis, p\aa \, tv\ae rs af et diverst ensemble af datas\ae t, som til sammen beskriver personligheder af millioner af mennesker i den vestlige verden. Berigelse n\ae r arketyper af variable som demografiske attributter, personlighedsfacetter, menneskelige v\ae rdier og adf\ae rd m\aa lt gennem smartphones, giver evidens for forskellige personlighedsstrategier hvoraf nogle r\ae sonnerer med kendte strategier fra literaturen. Vi sammenligner hvor godt individuel afstand fra personlighedsarketyper, Big Five tr\ae k samt principiel-komponenter af varians i sp\o rgeskema-svar kan modelleres ud fra adf\ae rdsdata. Simuleringen viser at afstand fra personlighedsarketyper kan modelleres bedst. P\aa \, baggrund af disse resultater, og underst\o ttet af litteratur i evolution\ae r psykologi, foresl\aa r vi at de fundne personlighedsarketyper kan betragtes som fundamentale adf\ae rdsstrategier.
