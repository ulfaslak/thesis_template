Out of the six archetypes which have been suggested based on the conducted analysis, all except the \host archetype appears to fit a strategy known from the literature. There are, however, a number of aspects which are important to the validity of the results which needs to be discussed. Some are scientific in nature and some regard the ethical implications.

\textbf{The Big Five model is "too crude" to capture fine-grained strategies}.
The Big Five factors result from factor analysis based in as many as 17.000 personality descriptive adjectives (English language), each of which was encoded into language arguably to fill some hole in comprehension. The model then lumps together all these words, into five factors that admittedly comprise a whole range of facets, but at the same time disregards much of the variance inherent to the language. For this reason, very fine-grained personality differences cannot be explained. For example, there is no trait in the Big Five model that successfully captures a persons psychopathic tendencies to form close relationships with people for the sake of personal gain. Trying to describe this behavior using the Big Five model is like trying to grab a coin with a boxing glove. As such, the strategies which are suggested in this work, are never less crude than the model they were inferred from.

% \textbf{The strategies are broad}.
% This relates to the above statement.
% Comparing to strategies treated in e.g. game theory, the ones suggested in this study are fairly broad spectered. This is to say, a strategy such as the \host could comprise any number of otherwise vastly different individuals. There might even me a few psychopaths in there. As such it shows personality strategies to be very unspecific constructs. Just because one knows the personality of another person their behavior might still be a mystery.

\textbf{Archetypes are not real people}.
There is an important notion to be made about the nature of these archetypal personalities. When presenting this work to people who are not familiar with it, and in particular when showing people the archetypes they have a strong tendency to quickly identify with one. Why people do this remains an open question but a speculation is that people may find it easier to understand things such as this by projecting their own self-image and feelings onto it. However, while people tend to do this, no one can really be described by just one archetype. In fact, by a geometrical argument (and as can be seen from Figure \ref{fig:consensusArchetypes}.b), no one is very close to an archetype - we are all described by \textit{all} archetypes at the same time. That is to say, since the archetypes were derived from the point distribution of personalities in Big Five space using the PCHA algorithm run with some allowed slack, points are generally closer to the middle of the simplex than they are to the archetypes.

\textbf{People are not necessarily stuck with a strategy}.
This study makes no claims about human nature beyond the strategies which can be inferred from the data. That is to say, whether a person can change strategy over a short period of time, or instantly depending on context, this study does not know. However, it does say that no person can fully assume more than one strategy \textit{at a time}. Picturing the Pareto front in 3D trait-space, this makes sense - a point can only be in one location at a time. Yet, the result that the \wildcard archetype tends to be far older than e.g. the \hippie suggests that people do change strategies over a lifetime.

% \textbf{Strategy fits niche and niche fits strategy}.
% The question of causality in terms of how strategy finds a niche is important.
% One may ask whether people with a certain strategy is drawn towads a particular niche or whether the niche people find themselves in shapes their strategy.
% The answer should be that it goes both ways.
% Personality psychology teaches that personality traits are 30-50\% inheritable which means that up to 70\% of a persons personality is adapted.
% As such there is a component of ones personality that draws one towards the most appropriate niche, 
% One may ask whether the \achiever archetype tends to be unemployed because it dislikes other people, or dislikes others because it tends to be unemployed.
% People are therefore both in control and not in control over their personalities, and this gives rise to the important notion that a persons niche 

\textbf{The type of people who take surveys on the Internet may not represent a general population}.
This very artifact may be what causes the distribution of traits to be so different between datasets that were collected online and per phone, as shown in Appendix \ref{app:BFdist}. However, it is unknown exactly how severe this biases the data, or if it biases it at all.

\textbf{Archetypes separate data better than traits do, but is this really so surprising?}
This short discussion would have been replaced by a simulation if time had permitted it. However, since that is not the case a few remarks on the result that archetypes are better prediction targets than BFTs and QI-PCs are due. First: Maybe it's not so strange? Archetypes are geometrical extremes and the data is not well-seperated so possible the best way to distinguish points is my learning which their nearest archetype is.

\textbf{Confirmation bias may have influenced the analysis}.
We are all humans. And as humans analyzing humans we are constantly at risk of making analysis choices that are under bias of this fact. So while care has been taken to withdraw excitement over preliminary results and "read too much into things", it cannot be rejected that some scientifically unhealthy humanity has seeped into the interpretations of the results.