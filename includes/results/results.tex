The research objective is to find fundamental personality archetypes and understand which behavioral strategies they correspond to in terms of the their personality facets, demographic attributes, values and correlated behaviors.
As stated in the introduction this thesis documents the first efforts in an ongoing research project, and as such the above goal is bold and will not be exhaustively satisfied.
The results do however offer clues towards \textit{which} evolutionary behavioral strategies can be inferred from data.

The analysis takes the following trajectory:
Six archetypes are found to emerge consistently across multiple datasets.
For each, values, personality facets, personal attributes and behavior is inferred.
The features of each archetype differ in fundamental ways and provide clues towards strategies for performing life's many tasks.
These are speculated to be: (1) \achiever (power/achievement driven, emotionally stable, unsympathetic), (2) \host (achievement driven, highly social, emotionally unstable), (3) \wildcard (conformist, benevolent, social, emotionally stable), (4) \loyalist (narrow-minded, loyal, orderly), (5) \hippie (unconformed, undutiful, unorganized, unworried, emotionally stable) and (6) \follower (antisocial, depressed, worried, emotionally unstable).
Note that for the sake of clarity these strategy labels are used throughout the analysis before they are given fair justification.

The analysis is separated into parts that each has a single premise.
Each motivates the next and together they discover the strategies.
The first part presents the analysis leading to the discovery of \textit{consensus archetypes}, a relaxed interpretation of an archetype.
The second part investigates the types of personal attributes that are enriched near each consensus archetype, and which stable behavioral indicators correlate with distance from the archetypes.
The third part investigates the predictive power of the consensus archetypes compared to raw BFTs and PCs of BF questionnaire items (QI).
Technical details about the analysis are explained in Section \ref{subsec:methods:analysis}.