From Figure \ref{fig:consensusArchetypes}.a it is observed that some archetypes share very similar trait values.
Comparing e.g. the \host and \follower archetypes they only differ on the \EXT trait.
Does this mean that the two may correspond to the same behavioral strategy? No.
In the following the differences between archetypes are investigated by studying how facets, values, demographic attributes and behaviors are enriched/depleted\footnote{Herinafter just \textit{enriched} to simplify the language.} for- or correlate with distance from the archetypes.
Enrichment analysis is used because it provides an intuitive way of interpreting whether a variable is influenced by distance from an archetype, and correlation analysis is used for behavior where there are not enough datapoints to provide statistically significant results using enrichment analysis (see Section \ref{subsubsec:ParTIptTwo}).
It is important to stress, however, that either of these approaches inform only about statistical \textit{trends} and not absolute differences.
For example, as was stated in Section \ref{subsubsec:behaviorAndPersonality} there is a limit to how well personality and behavior can correlate, empirically fixed around 0.3.
And while the same rule may not necessarily apply for demographic attributes, values and particularly not facets (for reasons explained later), it still raises the important notion that no relationship between \textit{anything} and personality is absolute, because after all (as was raised as critism in Section \ref{subsubsec:behaviorAndPersonality}) the best available method can only explain 56\% of personality variance.
The results presented in the following are therefore only \textit{indicative} and their implications are strictly useful for supporting claims that strategies exist and suggesting what those might then be.
Since the analysis yields many results, this section presents them in summary form.
The full collection of results is presented in Appendices \ref{app:questionnaireEnrichmentTable}-\ref{app:behaviorCorrelationTable}.

% \textbf{Facets and values reveal differences between archetypes}.
% Figure 
% Appendix \ref{app:questionnaireEnrichmentTable} and \ref{app:valuesEnrichmentTable} present all significant enrichments of QIs from the SPI inventory and BHVs, respectively, where rows have been permuted to illustrate for which archetypes maximum enrichment occurs.
% It is noted from Figure \ref{fig:firstEnrichmentsQuestionnaireValues} that there are stronger enrichments for QIs than for BHVs.
% While care has been taken to avoid direct circular enrichment, it appears from the strong QI enrichments that it still occurs implicitly.
% The importance of this analysis is therefore not to highlight how strong the enrichments are, but rather how strong they are relative to each other.
% The combinations of QIs that are enriched for each archetype is not trivial and provides insight about facets.
% For example, when considering enrichment slopes in Appendix \ref{app:questionnaireEnrichmentTable} the \achiever archetype who has above average \OPE achieves that score not by having medium/high values for each QI but by consistently being low in items related to the appreciation for art, creativity and reflection, while being higher on items that relate to understanding things quickly, solving complex problems and knowing the answer to many questions.
% Both classes of facets relate to openness and intelligence, but are not necessarily correlated.

% \textbf{Demographic attribute enrichments indicate niches in society}.
% Categorical demographic variables are studied for enrichment near archetypes.
% Table \ref{tab:attributesTable} shows the most significant demographic attribute enrichments for each archetype.
% It appears that each fulfills a distinct niche in society.
% For the \achiever and \follower archetypes it is observed that enriched disciplines don't involve working with people, which is the opposite for the \host, \wildcard and \loyalist archetypes.
% Curiously, the same two groups also differ in their relationship status, indicating that one strategizes to rely on people while the other does not.
% All archetypes except for \host and \wildcard are enriched for unemployment, where \loyalist and \hippie archetypes are younger and students, and \follower is enriched for retirement which partially explains the unemployment trend.
% The \achiever archetype is, however, young as evidenced by the depletion of age and 'retired' job status, in which case the enrichment of unemployment could point to self-employment of various types.
% The 'smoking' and 'exercise' attributes might not say much about societal niche, but are interesting in themselves and when viewed together.
% The \wildcard archetype exercises and doesn't smoke, the \hippie archetypes smokes but doesn't exercise and the \host archetype both smokes and exercises.
% Smoking and exercise is somewhat contradictory, so why would anyone do both? The \host archetype is enriched for stimulation and hedonism which partially explains this observation, but it also indicates an ability to hold contradictory beliefs simultaneously, which is undoubtedly a useful feature for individuals whose behavioral strategy is to use social engagement to advance in hierarchies.



% Due to the age differences between the archetypes, age-dependent demographic variables such as education level and marital status are not used.


% \textbf{We can now paint a picture of the strategies}
% REWRITE TITLE TO PREMISE FORM.
% There are very clear differences between the archetypes just from the QIs and BHV enrichments.
% The \achiever archetype is an extreme individualist, who is emotionally disconnected and values power and disregards benevolence.
% The \host archetype is a talkative extrovert, highly emotional and values social stimulation and hedonism.
% The \wildcard archetype is conscientious and happy, and strongly values almost every BHV (except hedonism, self-direction and power), and in particular benevolence and conformity.
% The \loyalist archetype values tradition and conformity and is uninterested in abstract ideas and art.
% The \hippie archetype is carefree and messy, values hedonism and stimulation and cares little about security and achievement.
% Finally the \follower archetype is avoids social interaction, worries a lot and values none of the BHVs, and in particular not social stimulation.