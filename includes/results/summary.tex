The analysis yields a large number of enrichments and correlations for each archetype, of which an excerpt is visualized in Figure \ref{fig:firstEnrichmentsQuestionnaireValues}, a shortened overview is shown in Table \ref{tab:facetsValuesAttributesTable}, and elaborate presentation is given in Appendices \ref{app:questionnaireEnrichmentTable}-\ref{app:behaviorCorrelationTable}. 
Here, each archetype is presented in prose highlighting their individual characteristics.
Each is examined and proposed a behavioral strategy with reference to existing literature when possible (primarily \textit{Evolutionary Psychology} (2014) by Workman et. al.\mcite{workman2014evolutionary}).


\subparagraph*{Self-concern}
The \achiever archetype is unemotional and mostly interested in power and achievement.
Occupied in fields which commonly don't involve working directly with people, whereas the depleted professional disciplines are the reverse.
Lacks sympathy, tends to exploit others for own ends and looks down on others.
Has a high degree of control through emotional stability and is hard to offend.
Highest valued BHVs other than power and achievement are stimulation and hedonism and least valued BHVs are benevolence, conformity and tradition.
Tends to be slightly younger (\textasciitilde5 years younger than furthest bin), particularly not retired individuals, not committed in relationships and has a tendency to exercise often.
Comprises many unemployed individuals.
There were found no significant correlations between distance from this archetype and behaviors measured through smartphones.

	\subparagraph{\textnormal{\textit{Discussion}}} 
	These are traits that are often prevalent in individuals with antisocial personality disorders (APD) and are typically lumped together using the term \textit{psychopathy} by evolutionary psychologists\mcite{workman2014evolutionary}.
	People suffering from psychopathy, i.e. \textit{psychopaths}, can use their manipulative tendencies to exploit others, gaining resources including sex, which is supported by studies that show ADP traits to correlate positively with an accelerated mating strategy\mcite{jonason2009dark, jonason2011mate}.
	Psychopathic traits are furthermore shown to lead to success in high-level business positions\mcite{hare2006psychopathy}, which the current results, however, don't support.
	Enriched disciplines are rather those of solitary nature, and as such it is just as reasonable to speculate this archetype to have a form of passive aggressive withdrawal strategy, that don't necessarily attract only attract psychopaths and individuals with other personality disorders.
	Furthermore, considering Figure \ref{fig:consensusArchetypes}.b distances to this archetype are typically small, which in other words mean that many people somewhat resemble this archetype, which is inconsistent with the converging agreement between psychologists that only up to 3\% of any larger population should suffer from sub-clinical psychopathy\mcite{kring2005abnormal}.
	As such it might be an artifact of a possible tendency in many humans to be competitive and, depending on context, feel emotions of contempt and superiority towards others, all of which may only surface when filling out a survey.
	The safest label for this strategy is therefore \achiever.

\subparagraph*{Social-achievement}
The \host archetype is highly social, talkative and lively, yet suffers from mood swings, insecurity and gets hurt easily.
Values are mostly stimulation, hedonism, power and achievement.
Has high level of employment and works high-pace jobs that directly involve people, such as sales, management and CEO positions.
Does not hold unskilled positions (food prep. worker, cashier) which resonates with the valuing of achievement and power.
Curiously both smokes and exercises, which is somewhat contradictory but can be explained by the high held value of stimulation and hedonism.
It may also indicate an ability to hold contradictory beliefs simultaneously, which is undoubtedly a useful feature for individuals whose behavioral strategy is to use social engagement to advance in hierarchies.
Maintains social relations through mobile phone by calling and texting a lot, socializes and moves around frequently both day and night and explores new places (visits places only once).
Typically makes long calls, sends more texts than receives and response-rate to texts or missed calls tends to depend on who the receiver is.

	\subparagraph{\textnormal{\textit{Discussion}}}
	The BFTs and enrichments for this archetype resonate with with known literature.
	Recall that this archetype is characterized by high levels of \OPE, \EXT and \NEU.
	\NEU is found to correlate positively with competitiveness and academic success for those who are resilient enough to cope with it\mcite{ross2001imposter, mckenzie2000neuroticism}, which agrees with the finding that this archetype works in fast-pace/high-status jobs such as CEO and operations manager.
	Provided that the goals of this archetype are mainly achievement and power (curiously similar to the \achiever archetype), is seems clear that this strategy focuses on using social interaction as a tool for acquiring resources and power.
	To the knowledge of this author there is no strategy treated in EP literature which strongly resembles this one.
	It can be argued that this strategy may attract individuals with multiple complex profiles, since it is a fairly universal drive to be highly social - and if one is also sensitive interacting with people may be a matter of earning popularity and using vulnerability to get ahead. This is discussed by Nettle\mcite{nettle2006evolution}.
	On a highly speculative note, given the occupation profiles, it might even attract psychopaths.
	However, for now it is simply named after its main facet \textit{social} and value \textit{achievement}.


\subparagraph*{Reciprocal altruism}
Happy, lively and social, well-tempered, not worried, trusting, hard-working and dutiful describe the most diverse set of enriched facets for the \wildcard archetype.
It highly values conformity, benevolence, stimulation and achievement, and is strongly enriched for relationship commitment and employment.
This archetype is more mature in age (closest bin is \textasciitilde10 years older than furthest bin), and tends to work in disciplines that have direct interaction with people.
Smoking is depleted and exercise is enriched, indicating self-discipline.
Has many friends outside of the university, frequently uses phone for communication, goes out at night, doesn't meet people in predictable weekly pattern, goes to many different places and tends to send the last text in chat conversations.

	\subparagraph{\textnormal{\textit{Discussion}}}
	EP build on the notion reciprocal altruism first introduced by Trivers\mcite{trivers1971evolution}. The idea, which is also mentioned briefly in Section \ref{subsubsec:individualDifferencesAndPersonality}, is that altruism may have developed as a mechanism for optimizing group fitness, or simply as a personal strategy for storing value by giving it away and expecting a return of similar value at a later time of greater need. Much about this archetype resemble this, and as such it is deemed to represent \textit{reciprocal altruism}.

\subparagraph*{Xenophobia-loyalism}
The \loyalist archetype is uncreative, simple-minded and slow, values conformity, tradition and security and neither self-direction, stimulation nor hedonism.
Works in health-care, office administration support and military, and particularly not computer science and arts.
Slight tendency to be younger (\textasciitilde3 years younger than furthest bin), non-smoker, committed in relationship and also often students.
Meets people in predictable weekly pattern, immediately responds to missed calls/texts, has long ongoing chats and shows up for class.

	\subparagraph{\textnormal{\textit{Discussion}}}
	Workman makes the remark that \textit{"Social psychologists have long known that the roots of xenophobia, or hatred of strangers, may be traced back to identifying strongly with one's own group and negatively stereotyping those of other groups."}. Going further back Darwin once noted: \textit{"The tribes inhabiting adjacent districts are almost always at war with each other [and yet] a savage will risk his own life to save that of a member of the same community."}\mcite{darwin1888descent}.
	Several enriched features of this archetypes resonate with characteristics commonly associated with xenophobia and loyalism. So the question of why this seemingly negative-sounding characterization should be an ESS for performing life's many tasks in such a way to increase fitness is fairly difficult to answer. Possibly this is remanence of a time where we lived in small tribes and had to react strongly to outsiders to survive, as noted by Darwin.
	However, others suggest that the ability to perceive sharp distinctions between "us and them" group members is crucial to the formation of lasting coalitions and is therefore an adaptive strategy\mcite{krebs1997social}.

\subparagraph*{Freeriding}
The \hippie archetype is undutiful, disorderly, not hard-working, breaks rules, unworried, emotionally stable and does not plan ahead.
Only values stimulation and hedonism, and particularly doesn't value security, achievement, tradition, power nor benevolence.
Professional disciplines that are enriched near the archetype include performance and visual arts, cultural studies, language and literature studies as well as communications and social science, while there is depletion for business and medicine/health related disciplines.
This archetype is younger (\textasciitilde6 years younger than furthest bin and not retired), not committed in relationship, tends to smoke and not exercise, and is enriched for student and unemployed.
Makes short calls, doesn't meet people in predictable weekly pattern, is very nocturnal, interacts with strangers and bad at responding to missed calls/texts.

	\subparagraph{\textnormal{\textit{Discussion}}}
	Workman remarks:
	\textit{"The evolution of cooperation among non-kin is greatly compromised by the existence of freeriders. Freeriders reap the benefits of cooperation without paying the costs and thus place themselves at a competitive advantage by exploiting cooperators. Many (e.g. Tooby et al., 2006)\mcite{tooby2006cognitive} have argued that, given the inherent disadvantages of cooperating with freeriders, cooperation could not have evolved unless mechanisms for detecting and dealing with freeriders also evolved."}
	While the behavior of the freerider - other than its immediate strategy - are not examined in literature, this archetype could be a candidate, to help probe further into the understanding of this strategy which relies on others to do the heavy lifting.

\subparagraph*{Withdrawal}
The \follower archetype is antisocial and reclusive, worried, depressed, suffers from mood swings and gets very emotional.
Values none of the BHVs, and particularly not stimulation, achievement, conformity and power.
Disciplines include arts, language and literature, computer science and natural sciences, and particularly not human interaction dependent disciplines in business, health-care and communication.
No significant enrichment of age, but enriched for unemployment and retirement which indicates that this strategy might attract older people.
Neither smokes nor exercises, which resonates with the low valuing of stimulation and hedonism, and is not committed in relationship.
Doesn't meet people at campus, not nocturnal, meets people in predictable weekly pattern, rarely calls/texts/looks at phone, has few contacts, stays at home.

	\subparagraph{\textnormal{\textit{Discussion}}}
	This archetype may also simply be called \textit{depression}. Models in literature view depression as adapted response to adverse conditions, which may help the individual accept defeat\mcite{wolpert2008depression}.
	Others attribute it more complexity, arguing that it is an involuntary strategy to signal yielding towards the surroundings such as to terminate unnecessary conflict (which would otherwise likely have been lost by the depressed)\mcite{price1994social}. If the latter is true, the label \textit{withdrawal} is appropriate.
