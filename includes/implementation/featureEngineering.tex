Feature engineering is carried out with focus on producing statistically independent behaviroal indicators, such as to capture as much of the behavioral spectrum with as little redundancy as possible. Views on importance of this is discussed in Appendix \ref{app:aNoteOnFeatureEngineering}. For extracting behavioral indicators (herinafter just \textit{indicators}) the Python package \textit{bandicoot} is used, and extended to handle multiple data-types. bandicoot is a call-detail-record (CDR) analysis package that offers phone companies and researchers to extract a comprehensive array of measurements of behavior based on text and call records\mcite{de2013predicting}. It works by loading records for individual users as comma separated values (CSV) which it expects to contain certain attributes such as \texttt{timestamp}, \texttt{interaction}, \texttt{correspondent-id}, into \texttt{Record} objects and creating \texttt{User} objects which contain their \texttt{Records} as parameters as well as other parameters inferred from the loaded records such as \textit{home} if the data contains location labels. The analyst then specifies whether to compute indicators for each day, week or month or the entire period comprising the data, whether to treat day and night or weekdays and weekends separately and finally which indicators to compute. Some of these rely on records being grouped into \textit{conversations} which are defined as dyad exchanges that are not delayed by more than some time constant (typically 30 minutes or 1 hour). Two extensions to bandicoot were made:
\begin{enumerate}
	\item Generalization of the the \texttt{Record} object to be any kind of data which only requires to have \texttt{timestamp} as an attribute.
	\item Modifying existing indicators to accept new data-types and adding new indicators that combine data from multiple channels.
\end{enumerate}
 
An example of (2) is that using multiple data channels allows for creation of indicators that show how much use their phone when they are social (channels: F2F and screen), how much they tend to go in groups rather than be in dyads (time overlap between F2F \textit{conversations}) and whether they tend to socialize with friends from the university \textit{outside of the university} (F2F and stop-locations). A full list of the indicators which are used is shown in Appendix \ref{app:fullListOfExtractedFeatures}, where it is marked which are added for this study and which already exist in the standard bandicoot version. The project code for the standard and extended versions of bandicoot is available, respectively, at:\\

\url{github.com/yvesalexandre/bandicoot}

\url{github.com/ulfaslak/bandicoot}\\

In this study, features were computed across the entire period since it was expected that the filtering criteria explained in Figure \ref{fig:data_activity} would provide sufficient behavioral consistency in the data.