As Figure \ref{fig:data_activity} shows, the activity of participants in the SensibleDTU study wasn't constant over the entire period, and although care was taken only to use data from the period of most activity (containing 526 participants), all participants cannot be expected to provide useful data that informs the analysis. Therefore two levels of filtering was set up.
The first level of filtering removes outliers using indicator activity thresholds based in careful assumptions about how people use smartphones. The aim is to filter out participants who were not actually using the phone. The assumptions are: (1) not everyone likes calling and texting, but everyone carries their phone with them on most days, (2) average over time people are within 1.5 meters distance of at least three people every day and (3) changes location at least twice daily. These conditions remove 84 participants from the analysis.
The second level of filtering uses kernel density estimation (KDE) for removing outliers\mcite{latecki2007outlier}. It employs a very simple scheme that computes kernel density (KD) (using the Python module \texttt{sklearn}) for each datapoint and removes those that fall below an empirically chosen threshold. This removes 30 participants from the analysis.
After both levels of filtering \textbf{412 valid datapoints remain}.