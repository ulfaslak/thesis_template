SAPA stands for 'Synthetic Aperture Personality Assessment' which is a survey collection method that feeds the respondent with semi-random questions from a large pool of questionnaire items from a variety of inventories, until the respondent stops answering. This creates a sparse dataset of response values, but for a large number of participants, covariance matrices between questionnaire items across people with different attributes (e.g. country/gender) can be computed to infer differences between groups.
The SAPA-Project is an effort by computational psychologist David Condon, which works as a web application allowing Internet users to take a personality test and get insights about themselves\mcite{condon2014}. In turn it is also a tools for researchers to gather personality data. Participants disclose various personal demographic information and respond to questions from three BF inventories: IPIP100, IPIP-NEO and SPI. The first two are internationally recognized pools of questionnaire items\mcite{donnellan2006mini}, while the last one is a product if Condon's thesis\mcite{condon2014}. Each one derives the BFTs using slightly different approached but largely produce the same factors. Using only responders who answered at least four items for each trait, the IPIP100 inventory has 77,685 valid datapoints, the SPI inventory has 2928 valid datapoints and the IPIP-NEO has 37,554 valid datapoints. 100\% of valid responders to the SPI inventory are also valid responders to the IPIP-NEO inventory and 47\% of valid responders to the IPIP-NEO inventory are also valid responders to the IPIP100 inventory.

An excerpt of the SAPA data is freely avaiable through the Harvard Dataverse at \url{https://goo.gl/vrX83G}.

