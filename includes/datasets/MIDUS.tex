Midlife Development in the U.S (MIDUS) was a longitudinal telephone/email survey study conducted in three waves in 1995, 2004/06 and 2013\mcite{brim2004midus}. Its goal was to create a dataset which could be analyzed to understand how various life variables affect each other. It successfully collected repeated questionnaire responses for 3294 randomly selected Americans, which, among many other metrics, included personality profiles measured with the BFM. For the BF items in the survey participants rated how well different words described them on a four tick scale.
In this study, BF datasets from each of the waves are used. MIDUS 1 has 6261 valid datapoints, MIDUS 2 has 3971 and MIDUS 3 has 2715. Importantly the MIDUS 2 and MIDUS 3 datapoints are measurements of participants who are also measured on in the previous years. For the intended purpose this is not thought to have any damaging influence on the results, but is still important to keep in mind. Considering Appendix \ref{app:BFdist} is can furthermore be seen that the \CON, \EXT, and \AGR traits do not distribute normally, but seem highly shifted towards higher values. This may be due to bias in the collection method which relied on an interviewer to collect the responses over telephone, or possible an artifact of the inventory used.