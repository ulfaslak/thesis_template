Behavior and personality are extensively studied topics in psychology. Mainly addressed in the sub-fields \textit{personality psychology} and \textit{trait theory} the scale of the body of knowledge accumulated in this area exceeds by far what is necessary to address in this thesis. As the title suggests, this section will discuss only topics which are deemed relevant to the main premise of this section: to establish reasonable cause for basing central parts of the analysis conducted in this project on the Big Five model. The primary literature resource throughout this section is Icek Azjen's textbook \textit{Attitute, Personality and Behavior} (2005, 2nd Ed.)\mcite{ajzen2005attitudes}, and when due citation is not appended to a qualified statement it is implicit that this is the source. It should be noted that in this thesis, the concept of \textit{attitute} -- as juxtaposed to \textit{behavior} and \textit{personality} by Azjen -- is not treated, although it could potentially be of interest to other researchers. Sec. \ref{subsubsec:behaviorAndPersonality} introduces behavior and personality as coupled constructs, and tests the limits of predictive modeling. Sec. \ref{subsubsec:lexicalHypothesis} presents the underlying hypothesis of trait theory -- the \textit{lexical hypothesis} -- and discusses its validity in light of criticism. Sec. \ref{subsubsec:dimensionsOfPersonality} gives a brief introduction to the development of trait theory and an in-depth account of the Big Five model. For the sake of completeness, a note on similar trait models of personality is also included.

% Point from analysis
Behavior exhibits stability over time but is highly context dependent.