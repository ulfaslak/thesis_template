In both psychology and everyday reasoning, human behavior is often explained by referencing underlying personality traits. When people cancel appointments they are called unreliable, when they make others laugh they are said to be charming, and when they are caught lying they are considered dishonest. As such, personality should be understood as that which you \textit{are} while behavior is that which you \textit{do}. More elaborate definitions characterize personality as \say{the set of emotional qualities, ways of behaving, etc., that makes a person different from other people}, and behavior as \say{the way a person or animal acts or behaves} or \say{the manner of conducting oneself}\footnote{"personality", "behavior". Merriam Webster.com. Merriam-Webster, 2011. Sat. 7 May 2016.}. Given these definitions it stands to reason that personality and behavior should be highly coupled constructs. Undoubtedly, the personality of an individual will affect its behavior, and in turn, its behavior will reveal information about its personality. This particular notion is of central concern to Icek Ajzen's \textit{Attitude, Personality and Behavior} (2005, 2nd Ed.)\mcite{ajzen2005attitudes}. The remainder of this discussion follows that discourse.

\begin{table}
	\centering
	\bgroup
	\def\arraystretch{1.7}
	\begin{tabular}{L{1.6cm}L{2.5cm}L{2.5cm}L{2.5cm}}
		\toprule
		\multirow{3}{1.6cm}{\textit{Nature of response}\vspace{0.2cm}}
		 & \multicolumn{3}{c}{\textit{Source of information about responses}} \\
		\cline{2-4}
		  & \textit{Observation} & \textit{Person} & \textit{Acquaintances }\\
		\hline
		Overt      & Motor acts, nonverbal cues, verbal response  & Self-reports of motor acts, nonverbal cues   & Peer-reports of motor acts, nonverbal cues   \\
		Covert  & Physiological responses, \textbf{personal device monitoring} & Self-reports of thoughts, feelings, needs, desires    & Peer-reports of thoughts, feelings, needs, desires   \\
		\bottomrule
	\end{tabular}
	\egroup
	\caption{\label{tab:responsesUsedToInferPersonality} Responses used to infer personality. As the columns indicate, behavioral responses may be recorded by observing the individual from the outside, by personal questioning or by interviewing acquaintances of the individual. For each of these approaches, researcher can either aim at measuring overt or covert behavioral responses. This table is an excerpt from {\normalfont [Ajzen 2005]}. The bolded text indicate information which has been added by this author.}
\end{table}

Ajzen argues that personality consists of latent, hypothetical characteristics which are not directly accessible, and can only be inferred from records of observable responses. Table \ref{tab:responsesUsedToInferPersonality}, which is an excerpt from Ajzen's book, establishes that response measurements can come from three sources: an observer, the subject itself, or someone who is acquainted with the subject. Responses can either be \textit{overt} - meaning directly observable acts/cues/expressions - or \textit{covert} - meaning not directly observed but inferred from interaction with the subject or from electronic measurements. It is important to note that responses need not be behavioral, but can also be answers to direct or indirect questions about the subject's personality, which is, for example, the case when inferring personality from questionnaires. In this study, measurements of behavior (for the sake of measuring behavior) are obtained through personal device monitoring (\textit{observation}/\textit{covert}) while measurements of personality are obtained through a questionnaire inventory (\textit{person}/\textit{covert}).

Trait theory is a field of inquiry that seeks to model personality as a set of traits (or characteristics) that each has influence on a range of behaviors. This, along with the earlier stated notion that personality and behavior are coupled, implies a degree of correlation between traits and behavioral manifestations, and so it is obvious to seek to construct predictive models between personality and behavior. In fact, a large parts of trait theory and an extensive amount of research, both historically and contemporary, addresses the modeling problem\mcite{mann1959review, behnke1974evaluation, paunonen2003big, de2013predicting, monsted2016phone}. Common for all, is that in no case has it been possible to produce statistically significant correlations between behavioral indicators and personalty traits that exceeded 0.3 in absolute correlation. This limit has been observed so frequently that is has come to be known as the "personality coefficient" (coined by Walter Mischel (1969)\mcite{mischel1969continuity}). Regardless of its broad acceptance, however, it is important to note that the personality coefficient is an empirical value.

Ajzen attributes the low correlation to effects governed by behavioral inconsistency, that is, even though someone might consider themself highly altruistic, they may not always act accordingly due to natural fluctuations in mood, circumstance, attitude, etc. Furthermore the notion of behavioral multi-determinism -  that traits express themselves through multiple behaviors, and reversely that behavior results from multiple trait expressions - is argued to be a contributing factor to low predictive validity between personality and behavior\mcite{ajzen2005attitudes}.

It is important to note that two variables with low correlation can still be statistically significant, and that often times this is the only thing which matters. In the case of this study, the purpose of correlating personality with behavior is not to build predictive models, but rather to study whether there is a significant relationship at all.