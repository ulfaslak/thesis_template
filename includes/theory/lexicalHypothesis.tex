The \textit{lexical hypothesis} (also: \textit{lexical approach}) is the fundamental assumtion that supports studies of personality founded in language. First applied by Sir Francis Galton (1884) to study adjectives in the English language, it remains the fundamental assumption that shapes trait theory today. It's intuition is exellently put by Lewis R. Goldberg (1990)\mcite{goldberg1990alternative}:

\begin{displayquote}
If a trait is truly important, it seems likely that over the course of the centuries, someone would have noticed it and given it a name.
\end{displayquote}
More accurately defined by Timothy D. Wilson (1988)\mcite{PER:PER2410020302}:

\begin{displayquote}
Those individual differences that are most salient and socially relevant in people’s lives will eventually become encoded into their language; the more important such a difference, the more likely it is to become expressed as a single word.
\end{displayquote}
It's a powerful assumption for the following reason: if the meaning of words are assumed to capture the true meaning the personality traits they describe, words themselves become a fabric for analyzing personality. As described further in Sec. \ref{subsubsec:dimensionsOfPersonality} this property is crucial when analyzing the structure of personality.

There are several arguments against the lexical hypothesis, whereof two stand out as most important: 
\begin{enumerate}
\item Since language itself is not developed by experts, the inherent ambiguity of words causes any model based on language sampling to, at best, reflect only a lay perceptions of the traits\mcite{cervone2007a, westen1996model}. 
\item Words can't accurately capture the spectrum of personality because some very important traits are tacit\mcite{argyle1973a}.
\end{enumerate}
It is important to emphasize that a model is never meant to capture the exact nature of a phenomenon. Rather, its purpose is to extract information about a system and help formulate pragmatic conclusions that accept the consequences of the underlying assumptions. For the purpose of this study, the lexical hypothesis is accepted as a valid theoretical foundation. 