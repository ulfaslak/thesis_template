Pareto Task Inference (ParTI) is a research paradigm developed at the Uri Alon lab to solve a number of problems related to evolution of morphology in various species. ParTI has been applied to understand systems in biology involving cancer cells\mcite{hart2015inferring}, morphology of ammonite shells\mcite{tendler2015evolutionary}, mass and longevity design of mammals\mcite{szekely2015mass}, and more\mcite{korem2015geometry, shoval2012evolutionary}. It is founded in principles relating to evolutionary trade-offs and Pareto optimality as explained in Section \ref{subsubsec:behaviorAndPersonality}, and combines these to provide a novel approach for inference of biological functions.

The goal of ParTI is to infer the \textit{evolutionary objectives} of an organism. Objectives can be anything from \textit{tasks} like cracking nuts and picking insects which a finch must perform well to remain evolutionarily fit, or even behavioral \textit{strategies} that a human can employ to perform all of life's evolutionarily important tasks in a way that maximizes fitness in a given niche. \textbf{The value proposition of ParTI is that it provides evidence of what these objectives are only by considering the traits of the organism in question}. ParTI works in two parts:
\begin{itemize}[leftmargin=.55in]
	\item [\textbf{\ref{subsubsec:ParTIptOne}}] Find archetypes, i.e combinations of traits that are optimized for specific objectives (\textit{Pareto}).
	\item [\textbf{\ref{subsubsec:ParTIptTwo}}] Infer objectives by studying the combinations of traits for each archetype (\textit{Task Inference}).
\end{itemize}

The central paradigm of ParTI exists in the first part, while the second part employs statistical methods to produce scientific results. In the following, each of these are examined.