This section is as much about personality psychology (PP) as it is about evolutionary psychology (EP), but that is only because EP is as much about any subfield of psychology as any subfield is about itself. EP boldly sets out to treat all parts of psychology from the fundamental assumption that all psychological processes result from Darwinian \textit{natural selection} - a mechanism which over time pressures "unfit" traits out of existence. It views the brain as an evolved organ designed by natural selection to guide the individual in making decisions that increase the chance of survival and reproduction. EP was conceived in 1992 following the publication of \textit{The Adapted Mind} by Leda Cosmides et. al., and is still considered an emerging field. Cosmides et. al. defines the discipline as \textit{"simply psychology that is informed by the additional knowledge that evolutionary biology has to offer..."}\mcite{barkow1995adapted}. Despite having earned a substantial amount of controversy and criticism (which even has a dedicated Wikipedia article to it\mcite{wikiCriticismEP}), EP has proven useful in explaining many concepts in e.g. language, cognition, intelligence, learning, adaptation, perception and personality\mcite{workman2014evolutionary, nettle2006evolution}.

Since the author of this thesis has not been given the appropriate schooling, in the field of psychology, to lecture any reader beyond the mere introductory aspects of EP, each part of this section is presented as a query rather than a block of knowledge. The first sections treat topics in PP mainly dealing with personality and behavior, how personality can be measured and what inherent limitations remain in the existing methods. Following this, personality is presented as treated by EP and the view of personality as a behavioral strategy is given due motivation. The material relating to PP relies mostly on Icek Ajzen's textbook \textit{Attitudes, Personality and Behavior} (2005)\mcite{ajzen2005attitudes}, while the material on EP uses Lance Workman and Will Reader's textbook \textit{Evolutionary Psychology} (2013)\mcite{workman2014evolutionary}.

% Evolutionary psychology (EP) is a theoretical approach that applies principles from Darwinian natural selection to the study of the mind. It views the brain as an evolved organ designed by natural selection to guide the individual in making decisions that increase the chance of survival and reproduction. The field was conceived in 1992 following the publication of \textit{The Adapted Mind} by Leda Cosmides et. al. The book defines the discipline as "simply psychology that is informed by the additional knowledge that evolutionary biology has to offer..."\mcite{barkow1995adapted}. Among many other topics, EP treats language, cognition, intelligence, learning and adaptation, perception and personality. This section will give an account of personality as it is treated in EP. It uses mainly the textbook \textit{Evolutionary Psychology} (2013) by Lance Workman and Will Reader as a source of information\mcite{workman2014evolutionary}, but also takes ques from Icek Ajzen's 'Attitudes, Personality and Behavior' to support some of the topics introduced from personality psychology and trait theory.

% What are the tasks:
% Renovating the Pyramid of Needs: Contemporary Extensions Built Upon Ancient Foundations

% Emergence of variation
% The fitness benefit of a strategy is thought to be \textit{frequency dependent} meaning that it either becomes less or more attractive when more individuals adopt it, and as such variation is perpetuated. Due to this constant change of tides, individuals are rarely optimized towards one strategy only, but typically adopt a mixture of multiple. Those who are, however, can be thought of as \textit{archetypes}.

% Recent work by Douglas Kendrick et. al., based in argument and literature review, suggests a revised version of Maslow's pyramid to capture \textit{motives} rather than \textit{needs}\mcite{kenrick2010renovating}, which reflect the type of tasks that are important to evolutionary fitness. Furthermore, various efforts make educated speculations and guesses about why personality traits have evolved\mcite{nettle2006evolution, workman2014evolutionary}. Given the abundance of personality data available to researchers it 